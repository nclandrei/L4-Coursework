\documentclass[paper=a4, fontsize=11pt]{scrartcl} % A4 paper and 11pt font size

\usepackage[T1]{fontenc} % Use 8-bit encoding that has 256 glyphs
\usepackage{fourier} % Use the Adobe Utopia font for the document - comment this line to return to the LaTeX default
\usepackage[english]{babel} % English language/hyphenation
\usepackage{amsmath,amsfonts,amsthm} % Math packages

\usepackage{lipsum} % Used for inserting dummy 'Lorem ipsum' text into the template

\usepackage{natbib}

\usepackage{sectsty} % Allows customizing section commands
\allsectionsfont{\centering \normalfont\scshape} % Make all sections centered, the default font and small caps

\usepackage{fancyhdr} % Custom headers and footers
\pagestyle{fancyplain} % Makes all pages in the document conform to the custom headers and footers
\fancyhead{} % No page header - if you want one, create it in the same way as the footers below
\fancyfoot[L]{} % Empty left footer
\fancyfoot[C]{} % Empty center footer
\fancyfoot[R]{\thepage} % Page numbering for right footer
\renewcommand{\headrulewidth}{0pt} % Remove header underlines
\renewcommand{\footrulewidth}{0pt} % Remove footer underlines
\setlength{\headheight}{13.6pt} % Customize the height of the header

\numberwithin{equation}{section} % Number equations within sections (i.e. 1.1, 1.2, 2.1, 2.2 instead of 1, 2, 3, 4)
\numberwithin{figure}{section} % Number figures within sections (i.e. 1.1, 1.2, 2.1, 2.2 instead of 1, 2, 3, 4)
\numberwithin{table}{section} % Number tables within sections (i.e. 1.1, 1.2, 2.1, 2.2 instead of 1, 2, 3, 4)

\setlength\parindent{0pt} % Removes all indentation from paragraphs - comment this line for an assignment with lots of text

%----------------------------------------------------------------------------------------
%	TITLE SECTION
%----------------------------------------------------------------------------------------

\newcommand{\horrule}[1]{\rule{\linewidth}{#1}} % Create horizontal rule command with 1 argument of height

\title{
\normalfont \normalsize
\textsc{University of Glasgow} \\ [25pt] 
\horrule{0.5pt} \\[0.4cm] % Thin top horizontal rule
\huge Safety Critical Systems AX Report (H)\\ 
\horrule{2pt} \\[0.5cm] % Thick bottom horizontal rule
}

\author{Andrei-Mihai Nicolae\\2147392n@student.gla.ac.uk} % Your name and email

\date{\normalsize\today} % Today's date or a custom date

\begin{document}

\maketitle % Print the title

\tableofcontents % Table of contents

%----------------------------------------------------------------------------------------
%	Introduction
%----------------------------------------------------------------------------------------

\section{Introduction}

\par
Artificial Intelligence has been one of the hot topics of this decade as it is surrounding our lives more as days go by. We can see it in objects ranging from aircrafts to smart home controllers. \\

However, as AI has become so powerful, it may come with a cost for the majority of us. Because of the large number of risks when letting Artificial Intelligence and Machine Learning robots/devices perform certain actions for us (\cite{ai-risks}), we need to devote a huge amount of effort in designing better safety-oriented architectures, well-crafted and thorough testing as well as regular check-ups and revisions. Therefore, we need to see a considerable increase in tools that assess and mitigate risks when introducing AI into systems of any type. Such a tool will be discussed in further detail in this report.

%------------------------------------------------

\subsection{AI-Driven Technologies}

\par
The importance of introducing AI into a field with safety as a top priority is crucial. Because there is no human involved, the goal of the whole AI community is to let the machines actually take our place in performing certain actions, so that our tasks would be simplified. As good examples where it has become more and more developed, here is a list of examples:

\begin{itemize}
	\item Transportation (driverless cars, subways)
	\item Game playing machines (Deepmind's Go playing machine that beat the en-titre champion) \citet{deepmind}
	\item Medical robotics
	\item Manufacturing machines
	\item Education
\end{itemize}

\par
As the list can go on, we can see how AI spans throughout most of the major aspects of our lives, thus the need of careful monitoring its development.

%------------------------------------------------

\subsection{How AI Can Go Wrong}

\par
Going past the many fields driven by Artificial Intelligence nowadays, we need to also take a close look at how many technologies have proven to be very prone to failure. \\

\par
One interesting case is something that happened only a few months ago with an Uber driverless car going through a red light in front of San Francisco's Museum of Modern Art (\citet{uber-red-light}).

\par
As it was recorded on camera, the car just rushes through a busy street on red light. This is a clear sign of how developers are not placing enough testing and robust checks before launching such a safety critical systems into an open environment. \\

\par
We can see that the cause of the previous example could have been harmful for us humans. However, there have been cases where AI was involved and it was even deadly. Such an event is the killing of a Volkswagen employee who was grabbed and killed by one of the manufacturing robots in the plant (\citet{volkswagen}). Moreover, a robot in one Silicon Valley mall struck a child on the head by mistake (\citet{knightscope}). \\

\par
In conclusion, after discussing various developments in the Artificial Intelligence world and how these safety critical systems can fail drastically, a tool for assessing and mitigating such risks will be presented in the rest of this report.

%----------------------------------------------------------------------------------------
%	Tool
%----------------------------------------------------------------------------------------

\section{Tool}

\par
After extensive research, I came to the conclusion that for such an application the best technique that should be used is Model Checking adapted specifically for AI systems. \\

%------------------------------------------------

\subsection{Reasons behind Choice}

\par
Firstly, it's needed to be pointed out that various techniques (e.g. Fault Tree Analysis, Effects and Criticality Analysis) work as well for some sub-fields of AI-driven systems, but I believe that Model Checking is eventually the optimal choice. This is because of various reasons which will be exposed below.

\subsubsection{AI is Young}

\par
Because of the relatively young age of AI and ML in mainstream technology, there is quite a significant amount of improvements needed to be implemented. As shown before, there are many cases where they failed and lead to possible catastrophic outcomes. \\

\par
Model Checking is a solution to the problem: having a model system, check automatically (and exhaustively) if that particular system meets some given specification. I believe this is a very good approach to the tool needed to be implemented because the model checking applies to finite state systems (\citet{wiley-encyclopedia}). As AI is young, we need to make sure (even at the cost of time and not so optimal design decisions) that the system behaves correctly and gives the correct output regardless of the situation.

\par
As such, a technique such as Model Checking would automatically check all possible combinations and determine whether the system is prepared to use AI/ML safely or not.

\subsubsection{State of the Art Usage of the Model}

\par
Mention here about NASA and SPIN (slides are in Downloads folder)

\subsubsection{Large Number of Tools}

\par
Talk about the large number of tools available for implementing model checking

%------------------------------------------------

\subsection{Case Study}

\par
Add images here and talk about a detailed use case (linking to other studies showing that model checking is used throughout the world)
%------------------------------------------------

\subsection{Case Study}

\subsection{Example of list (enumerate)}
\begin{enumerate}
	\item First item in a list 
	\item Second item in a list 
	\item Third item in a list
\end{enumerate}

%----------------------------------------------------------------------------------------
%	Evaluation
%----------------------------------------------------------------------------------------

%----------------------------------------------------------------------------------------

\section{Evaluation}

%----------------------------------------------------------------------------------------
%	Results
%----------------------------------------------------------------------------------------

%----------------------------------------------------------------------------------------

\section{Results}

%----------------------------------------------------------------------------------------
%	Conclusion
%----------------------------------------------------------------------------------------

%----------------------------------------------------------------------------------------

\section{Conclusion}

\bibliographystyle{plainnat}
\bibliography{bibliography}

\end{document}

