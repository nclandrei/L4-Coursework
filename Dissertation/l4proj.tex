\documentclass{l4proj}

\usepackage{url}
\usepackage{fancyvrb}
\usepackage[final]{pdfpages}


\begin{document}
\title{Algorithm Animator}
\author{Andrei-Mihai Nicolae}
\maketitle

\begin{abstract}
Understanding algorithms is both very common and hard for developers in general,
regardless of their level of expertise. Even the fundamental ones, such as Dijkstra's algorithm
for finding the shortest path between two nodes in a graph, are quite complicated to grasp. Many 
studies show that visualizing an algorithm and its steps make understanding it much easier. In
this report, we will present an Algorithm Animator built specifically for solving this problem 
in a modern, responsive and efficient manner. Among others, we will also show why certain design decisions (e.g. making it a native
desktop app instead of a basic jar, using material design for the user interface), the implementation choices and the
evaluation results make this tool a viable option for software engineers when it comes to learning different kinds of 
algorithms.
\end{abstract}

\educationalconsent

\tableofcontents
%==============================================================================
\chapter{Introduction}
\label{intro}
\pagenumbering{arabic}



\section{Aims}

\section{Motivation}

\section{Contributions}

\section{Report Content}
The rest of the report will analyze the background of animators and why they were proven useful, as well as cover all the steps in gathering requirements, designing, implementing, testing and evaluating the tool. 

\begin{itemize}
\item Chapter~\ref{background} covers work related to the purpose of algorithm animators and why they are useful 
\item Chapter~\ref{requirements} goes into how the problem was analyzed and what requirements were gathered through
	project meetings and discussions with Algorithmics students.
\item Chapter~\ref{design} explains the design decisions behind the tool and illustrates various lessons learned and
	problems faced along the way.
\item Chapter~\ref{implementation} goes into the implementation details of the animator.
\item Chapter~\ref{testing} show how extensive unit, integration and other types of testing (e.g. smoke, end-to-end)
	were undergone and why they were essential to the development of the application.
\item Chapter~\ref{conclusions} details the overall results of the project.
\end{itemize}

%==============================================================================

\chapter{Background}
\label{background}

\section{Related Work}

%==============================================================================

\chapter{Requirements}
\label{requirements}

\section{Problem Analysis}

\section{Requirements Gathering}

\section{Functional Requirements}

\section{Non-Functional Requirements}

%==============================================================================

\chapter{Planning}
\label{planning}

\section{Agile}

\subsection{Kanban Board}

\subsection{Issues \& Bug Tracking}

%==============================================================================

\chapter{Design}
\label{design}

\section{Architecture}

\subsection{EDA (Event-driven Architecture)}

\section{Native Desktop App vs. Jar}

\section{Electron}

\section{Vis.js}

\section{Material Design}

\section{Compromises}

%==============================================================================

\chapter{Implementation}
\label{implementation}

\section{Project Structure}

\section{JavaScript and Multi-Threading}

\section{JS Animation Engine}

\section{Extra Features}

\section{Lessons Learned}

\section{Issues Faced}

%==============================================================================

\chapter{Testing}
\label{testing}

\section{Unit Testing}

\section{Integration Testing}

\section{Prototype Evaluation}

\section{Results}

%==============================================================================

\chapter{Conclusions}
\label{conclusions}

\section{Open Source}

\section{Project Roadmap}

\section{Final Thoughts}

\bibliographystyle{plain}
\bibliography{l4proj}
\end{document}
